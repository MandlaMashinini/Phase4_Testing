\documentclass{article}
\usepackage[utf8]{inputenc}

\title{Testing Phase}
\author{Minal Pramlall}
\date{May 2017}

\usepackage{natbib}

\begin{document}

\maketitle

\section{Non-Functional Requirements}

\begin{enumerate}
    \item Maintainability:\\
    \textbf{Mark: }
    10.\\
    \textbf{Comment:}
    Classes are separated well and allows easier access and retrieval to the object data.\\ \\
    
    \item Performance:\\
    \textbf{Mark: }
    10.\\
    \textbf{Comment:}
    Recognized that Postgres is the best performing database for this system, thus has better response times.\\ \\
    
    \item Scalability:\\
    \textbf{Mark: }
    10.\\
    \textbf{Comment:}
    Return methods allow for batch processing, returning a collection of objects at a time, as opposed to one a time.\\ \\
    
    \item Interoperability:\\
    \textbf{Mark: }
    10.\\
    \textbf{Comment:}\\
    This module runs with Java, similar to the other teams, and thus allows interchangeability with the other modules or the interfaces it can communicate with.\\ \\
    
    \item Usability:\\
    \textbf{Mark: }
    10.\\
    \textbf{Comment:}
    Enough class accessors and mutators have been provided to allow full use of the data objects.\\ \\
    
    \item Data Integrity and Security:\\
    \textbf{Mark: }
    10.\\
    \textbf{Comment:}
    Communication between classes leave little room for problems with data communication, GIS also has little to no interaction with any sensitive information so security measures need not be so strenuous.\\ \\
    
    \item Transparency:\\
    \textbf{Mark: }
    10.\\
    \textbf{Comment:}
    GIS module returns the locations and as much information relating to it, which makes the need for it to be as transparent as possible, which is delivered. Class objects provide access to all the fields and the methods themselves are named to clearly describe their function.\\ \\
    
    \item Documentation:\\
    \textbf{Mark: }
    10.\\
    \textbf{Comment:}
    All methods are well documented.\\ \\
    
    \item Reliability:\\
    \textbf{Mark: }
    9.\\
    \textbf{Comment:}
    Methods of this subsystem consistently return correct data, minus getting routes, which is - nevertheless - still mostly correct.\\ \\
    
    \item Availability:\\
    \textbf{Mark: }
    10.\\
    \textbf{Comment:}
    Methods are easily accessable and maintain consistency with return values.\\ \\
\end{enumerate}

\end{document}